\documentclass[a4paper,10pt]{article}
%\documentclass[a4paper,10pt]{scrartcl}

\usepackage[utf8]{inputenc}

\title{DLX architecture}
\author{Flavio Tanese}
\date{2017/5/25}

\begin{document}
\maketitle

\section{Memory access}
 This custom DLX uses a Von Neumann architecture with data and instructions on the same physical memory.
 
 All memory accesses are aligned, meaning that fetching a half-word requires 1 less address bit than fetching a byte
 (as the LSB is forced to 0 and can be hardwired to ground somewhere else). Still, since the spare bit would be unused
 and additional hardware would be required to pass a different number of bits to the memory access unit, it was decided
 that this property would not be exploited.
 
 Fetched data are stored in a data cache, while instructions are stored in a separated instruction cache thus
 virtualizing a Harvard architecture.
 
 The architecture is big endian with 32-bit registers: operations work on full 32 bits even if the operands would
 require less. Loading a smaller operand will result in sign extension to get its size to 32 bits. Double precision
 floating point is supported and requires two floating point registers.
 
 Table \ref{tb:MTI} contains the full list of memory transfer instructions.
 
 \begin{table}
  \begin{tabular}{|cc|}
   \hline
   Instruction & Meaning\\
   \hline
   LB, LBU, SB & Load Byte, Load Byte Unsigned, Store Byte\\
   LH, LHU, SH & Load Half word, Load Half word Unsigned, Store Half word\\
   LW, SW      & Load Word, Store Word\\
    
  \end{tabular}
  \label{tb:MTI}
  \caption{Memory transfer instructions}
 \end{table}

\end{document}
